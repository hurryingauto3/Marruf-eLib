\documentclass[a4paper, 11pt]{article}
\usepackage[utf8]{inputenc}
\usepackage{graphics}
\usepackage{amssymb}
\usepackage{hyperref}
\usepackage{geometry}
\usepackage{tikzsymbols}
\geometry{margin=1in}
\hypersetup{
colorlinks=true,
linkcolor=blue,
filecolor=magenta,
urlcolor=blue,
}
\urlstyle{same}

\newcommand{\Maaruf}{\textit{Maaruf}}

\title{CS 353: Software Engineering \\ {\Large Final Project} \\ {\large Technical Constraints Document}}
\author{Team \Maaruf}

\begin{document}
\maketitle
\section{Team Members:}
Team \Maaruf is comprised of the following people:
\begin{enumerate}
    \item Ali Hamza \hspace{5mm} (ah05084)
    \item Maaz Saeed \hspace{5mm} (ms05050)
    \item Muhammad Usaid Rehman  \hspace{3mm} (mr04302)
\end{enumerate}
\vspace{2 mm}
\noindent Link to Github repository: \url{https://github.com/hurryingauto3/Marruf-eLib}

\section{Technical Constraints}


\begin{enumerate}
    \item The application will run on laptop and desktop machines that run Windows. The application will not be able to run on MacOS and Linux systems. 
    \item The application will contain two screens, an e-reader and a user library. 
    \item Ebooks will be stored on disk and will be accessed by the application when opened by a user to read. The application will not be relying on cloud storage, nor will it be able to access books stored on any cloud storage.
    \item The application will contain an import ebook feature which will store the metadata of the ebook and will add it to the user library. 
    \item The user library screen will let the user view metadata for each ebook. Users will be able to edit metadata as well (tentative). 
    \item The metadata for each ebook will be stored in a light database (\texttt{MongoDB} or \texttt{IndexedDB}). 
    \item The frontend will be created using \texttt{ElectronJS}. Electron has some issues with regards to performance overheads, but since the application is not that large, that will not be an issue for us. 
    \item The current scope of the application only supports PDFs as we plan to use a library called \texttt{pdf.js}.
    \item The backend will be made through \texttt{Node.js}.
\end{enumerate}

\end{document}

